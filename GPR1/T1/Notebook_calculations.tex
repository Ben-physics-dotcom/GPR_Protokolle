\documentclass{scrarticle}
\usepackage{imakeidx}
\usepackage{ragged2e}
\usepackage{setspace} % Um den Zeilenabstand zu ändern.
\usepackage{gensymb}
%\usepackage{authblk}
% \usepackage{minitoc} % for the chpaters
\usepackage{wasysym}
%\usepackage{SI}
\usepackage{array} % Verwendung von Matrizen
\usepackage{booktabs} % Schöne Tabellen beziehungsweise sie sehen damit professioneller aus.
\usepackage{tabulary} % Ähnlich wie tabularx, ermöglicht aber das ändern der Ausrichtung der Spalten.
\usepackage{tabularx} % Tabellen mit automatischen Zeilenumbruch.
\usepackage{enumitem}
\usepackage{physics}
\usepackage[T1]{fontenc}% fontec und inputenc ermöglichen
\usepackage{graphicx}%Für Grafiken
\usepackage{rotating} % lässt Grafiken rotieren
\usepackage{mathtools}% mathematische Werkzeuge
\usepackage{amsmath}% Mathetools
\usepackage{amsfonts}% Mathetools
\usepackage{amssymb}% Symbole wie Natürliche Zahlen
\usepackage{geometry}
%\usepackage{bibtex} 
\usepackage{tablefootnote}% Fußnoten in Tabellen
\usepackage{float}% für eingebundene Bilder
\usepackage{fancyhdr} % Seiten schöner gestalten, insbesondere Kopf- und Fußzeile
\usepackage{ulem} 
\usepackage{dcolumn}% Align table columns on decimal point
\usepackage{bm}% bold math
\usepackage[ngerman]{babel} % Worttrennung nach der neuen Rechtschreibung und deutsche Bezeichnungen. babelfunktion wird wegen Literatur gebraucht.
\usepackage{subfloat} % Was macht diese Packet?
\usepackage{caption} % Unter-/Überschriften für Bilder, Grafiken und Tabellen
\usepackage{subcaption}
\usepackage{txfonts}
\usepackage{titling}% Titel
\usepackage[style=alphabetic]{biblatex} %biblatex mit alphabetic laden. alphbetic=Zitationsstil
\usepackage{bookmark}
\usepackage[printonlyused]{acronym}
\usepackage{amsthm}
\usepackage{pdfpages}
\usepackage{tikz}
\usepackage[siunitx,americanvoltages, europeanresistors,americancurrents]{circuitikz}
\usepackage{listings}
\usepackage{abstract}
\usepackage[per-mode = fraction]{siunitx}
\usepackage{hyperref}
\newcommand{\R}{\mathbb{R}} % reelle Zahlen
\newcommand{\N}{\mathbb{N}} % natürliche Zahlen
\newcommand{\C}{\mathbb{C}} % komplexe Zahlen
\newcommand{\Q}{\mathbb{Q}} % rationale Zahlen
\newcommand{\Z}{\mathbb{Z}} % ganze Zahlen
\newcommand{\F}{\mathbf{F}} % Kraft
\newcommand{\E}{\mathbf{E}} % Energie
\newcommand{\V}{\mathbf{v}} % Geschwindigkeit
\newcommand{\B}{\mathbf{B}} % magnetischer Fluss
\newcommand{\J}{\mathbf{j}} % Stromdichte ?
\newcommand{\D}{\mathbf{D}} % elektrische Induktion
\newcommand{\HH}{\mathbf{H}} % magnetische Feldstärke
\newcommand{\M}{\mathbf{M}} % Magnetisierung
\newcommand{\p}{\mathbf{P}}
\newcommand{\rr}{\mathbf{r}}
\newcommand{\vp}{\varphi}
\newcommand{\ve}{\varepsilon}
\newcommand{\vcc}[1]{\left(\begin{matrix}#1\end{matrix}  \right)}
\newcommand{\m}[1]{\left\lbrace #1\right\rbrace}
\newcommand{\los}{\noindent\textbf{Lösung}:}
\newcommand{\rang}[2]{\text{Rang}(#1)=#2}
\newcommand{\vpe}{\frac{1}{4\pi\ve_0}}
\newcommand{\qvpe}{\frac{q}{4\pi\ve_0}}
\newcommand{\geg}{\ac{geg.}}
\newcommand{\ges}{\ac{ges.}}

\newcommand{\kommando}[1]{$\backslash$\textit{#1}}
\newcommand{\com}[1]{$\backslash$\textit{#1}$\left\lbrace\ldots\right\rbrace$}
\newcommand{\Com}[2]{$\backslash$\textit{#1}$\left\lbrace #2\right\rbrace$}
\newcommand{\NeuKommando}[2]{$\backslash \textit{#1} \left\lbrace \backslash \textit{#2}\right\rbrace$}
\newcommand{\latex}{\LaTeX $\;$}


% si unitx
\DeclareSIUnit\litre{l}

\hypersetup{
	colorlinks=true,
	linkcolor=blue,
	filecolor=magenta,      
	urlcolor=cyan,
	citecolor=lime!50!black,
	filecolor=red
}
%\addbibresource{} %Bibliographiedateien laden
\addbibresource{bib.bib}

\geometry{a4paper, left=25mm, right=25mm, top=30mm, bottom=30mm}
\lhead{\thedate}
\rhead{GPR}
\lhead{\thetitle}
\pagestyle{fancy}

\usetikzlibrary{patterns}
\usetikzlibrary{3d}
\makeindex[title=Stichwortverzeichnis,intoc
,options= -s Index-Formatierung.ist
]
\author{Ben J. F.}
\allowdisplaybreaks

\lstset
{ %Formatting for code in appendix
    basicstyle=\footnotesize,
    numbers=left,
    stepnumber=1,
    showstringspaces=false,
    tabsize=2,
    breaklines=true,
    breakatwhitespace=false,
}

\title{Rechnungen zum T1-Versuch}
\begin{document}

\maketitle
	\tableofcontents
	
    \addsec{Abkürzungsverzeichnis}
\label{sec:abkuerzungsverzeichnis}

\begin{acronym}[Variation]
    \acro{Abb.}{Abbildung}
    \acro{allg.}{allgemein}
    \acro{AWP}{Anfangswertproblem}
    \acro{Bsp.}{Beispiel}
    \acro{bspw.}{beispielsweise}
    \acro{bzgl.}{bezüglich}
    \acro{bzw.}{beziehungsweise}
    \acro{CU}{Cauchy-Ungleichung}
    \acro{d.h.}{das heißt}
    \acro{de}{deutsch}
    \acro{DGL}{Differentialgleichung}
    \acro{dof}{degree of freedom}
    \acro{ED}{Elektrodynamik}
    \acro{eng}{englisch}
    \acro{erf}{Fehlerfunktion}
    \acro{etc.}{et cetera}
    \acro{EW}{Eigenwert(-e)}
    \acro{EV}{Eigenvektor(-en)}
    \acro{FPR}{fortgeschrittenen Praktikum}
    \acro{Fund.s.}{Fundamentalsystem}
    \acro{GD}{Gantt-Diagramm}
    \acro{geg.}{gegeben}
    \acro{ger}{german}
    \acro{Glg}{Gleichung}
    \acro{GPR}{Grundpraktikum}
    \acro{griech.}{griechisch/-[e,es,er]}
    \acro{HU}{Humboldt-Universität zu Berlin}
    \acro{i.A.}{im Allgemeinen}
    \acro{konst.}{konstant}
    \acro{ML}{Maschinelles Lernen}
    \acro{MS}{Microsoft}
    \acro{Nst}{Nullstelle(-n)}
    \acro{o.ä.}{oder ähnliches}
    \acro{Skript}{Vorlesungsskript}
    \acro{SI}{Le Système International d'Unités}
    \acro{STW}{Stichwörter}
    \acro{Theo}{Theoretische Physik}
    \acro{TSE}{\TeX$\;$Stack Exchange}
    \acro{urspr.}{ursprünglich}
    \acro{usw.}{und so weiter}
    \acro{u.a.}{unter anderem, und andere[s]}
    \acro{u.ä.}{und ähnliche[s]}
    \acro{ugs.}{umgangssprachlich}
    \acro{VL}{Vorlesung}
    \acro{VONS}{vollständiges, orthonormiertes System}
    \acro{Variation}{Variation der Konstanten}
    \acro{VSCode}{Visual Studio Code}
    \acro{vgl.}{vergleiche}
    \acro{z.B.}{zum Beispiel}
    \acro{z.z.}{zu zeigen}
\end{acronym}

\section{Rechnungen}
Quelle: \cite[\ac{vgl.}][15]{Muller.g}
\begin{align}
    c_w &= 4.1813 \unit{\kilo\joule\per\kilogram\per\kelvin}\\
    U \cdot I \cdot \Delta t &= (m_w \cdot c_w+ C_K) \Delta T\\
    \Delta T &= \frac{U \cdot I }{m_w \cdot c_w+ C_K}\cdot \Delta t\\
    a &= \frac{U \cdot I }{m_w \cdot c_w+ C_K}\\
    \Rightarrow C_k &= \frac{IU}{a}-m_w \cdot c_w
\end{align}
\noindent Aufgabe 1:
\begin{align}
    246 \unit{\milli\litre} \Rightarrow 0.246\unit{\kilogram}
\end{align}
\noindent Aufgabe 3:
\begin{align}
    242\unit{\milli\litre} \Rightarrow 0.242\unit{\kilogram}
\end{align}
\printbibliography
\end{document}
