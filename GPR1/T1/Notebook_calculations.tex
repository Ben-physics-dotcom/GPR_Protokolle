\documentclass{scrarticle}
\input{Settings/packages}
\input{Settings/commands}
\input{Settings/settings}
\title{Rechnungen zum T1-Versuch}
\begin{document}

\maketitle
	\tableofcontents
	
    \input{Settings/acronyms}
\section{Rechnungen}
Quelle: \cite[\ac{vgl.}][15]{Muller.g}
\begin{align}
    c_w &= 4.1813 \unit{\kilo\joule\per\kilogram\per\kelvin}\\
    U \cdot I \cdot \Delta t &= (m_w \cdot c_w+ C_K) \Delta T\\
    \Delta T &= \frac{U \cdot I }{m_w \cdot c_w+ C_K}\cdot \Delta t\\
    a &= \frac{U \cdot I }{m_w \cdot c_w+ C_K}\\
    \Rightarrow C_k &= \frac{IU}{a}-m_w \cdot c_w
\end{align}
\noindent Aufgabe 1:
\begin{align}
    246 \unit{\milli\litre} \Rightarrow 0.246\unit{\kilogram}
\end{align}
\noindent Aufgabe 3:
\begin{align}
    242\unit{\milli\litre} \Rightarrow 0.242\unit{\kilogram}
\end{align}
\printbibliography
\end{document}
